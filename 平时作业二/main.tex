%! Tex program = xelatex   
\documentclass{homework}
\usepackage{xeCJK}
\usepackage{amsmath}
\usepackage{booktabs} %表格
\usepackage{listings}
% \setmainfont{Times New Roman}
\setCJKmainfont{Kaiti SC}
% \setCJKfamilyfont{song}{Songti SC}
% \renewcommand{\baselinestretch}{1.5} %行间距
\author{朱浩泽 1911530}
\class{计算机系统设计}
\date{\today}
\title{\Large{平时作业二}}


\graphicspath{{./media/}}

\begin{document} \maketitle

\question \large{代码段一:
\begin{lstlisting}[language = c++]
int a = 0x80000000;
int b = a / -1; 
printf("\%d\n", b);
\end{lstlisting}

运行结果为-2147483648
~\\

\ \ \ \ \ \ \ \ \ \ \ \ \ 代码段二:
\begin{lstlisting}[language = c++]
int a = 0x80000000;
int b = -1;
int c = a / b; 
printf("%d\n", c);
\end{lstlisting}

运行结果为“Floating point exception”,显然CPU检测到了异常。

请问,为什么显示是“浮点异常”呢?
}

% \normalsize
~\\
答:第一段代码中a直接除以-1,可以将负整型常量转换为无符号类型来处理,所以最终所得的输出是-2147483648。
处理代码段二时,b是一个有符号的变量,a/b计算所得的有符号数是-0x80000000即-2147483648,对于2147483648来说已经占了32位,此时如果再加上符号位则超出了int的32位表示范围,而编译器又严格参照IEEE规定,认为“溢出”的情况属于出现了“不存在的数字”,或者说是产生了未定义的无穷大的数,这样的情况则会被认为是产生了浮点数异常。


~\\
~\\
~\\
\question \large{为什么整数除0会发生异常而浮点数除0不会?}

% \normalsize

~\\
答:使用整数除0会发生异常使程序终止,而浮点数除0会显示出为无穷大或NaN之类的输出。这是因为,浮点数的表示形式中,存在着表示无穷大的方式,而整数却不存在这种表示方式,所以抛出异常。

\end{document}
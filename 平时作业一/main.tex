%! Tex program = xelatex   
\documentclass{homework}
\usepackage{xeCJK}
\usepackage{amsmath}
\usepackage{booktabs} %表格
% \setmainfont{Times New Roman}
\setCJKmainfont{Kaiti SC}
% \setCJKfamilyfont{song}{Songti SC}
% \renewcommand{\baselinestretch}{1.5} %行间距
\author{朱浩泽 1911530}
\class{计算机系统设计}
\date{\today}
\title{\Large{平时作业一}}
% \address{Bayt El-Hikmah}

\graphicspath{{./media/}}

\begin{document} \maketitle

\question \large{PPT翻页从按下翻页键到显示器显示下一页,中间经过了哪些过程和重要环节}
\normalsize
~\\
答:首先鼠标点击翻页按钮,会有一个电信号传到CPU中的控制器,操作系统捕获到该消息,将它放到消息队列里处理,控制器会进行系统调用(这时,有可能会感觉到翻页并没有立马进行,这可能是因为进程在内存中一直处于阻塞的状态,即该操作处于消息队列优先级较低的位置),在其他调用完成后,该操作便开始执行。然后操作系统便会调出相应的对应于翻页的文件,其过程文件会被驻入内在,等待CPU中处理器对内存中数据进行查找、执行撤下原先的 PPT 图片和其他资源、换上刚找到的下一页的资源、更新显示器等一系列操作。当执行完毕后,结果又会返回到内在,由控制器将其交给显卡中的缓存区,由显卡做出图像运算,并将翻页动作显示在屏幕上。
~\\
~\\
\question \large{用户CPU时间与系统响应时间哪个更长?举例说明}

\normalsize
~\\
答:用户CPU时间指的是执行用户指令所用的时间,就是用户的进程获得了CPU资源以后,在用户态执行的时间;系统响应时间是计算机对用户的输入或请求作出反应的时间,包含了执行时间和等待时间。\\
系统响应时间大于用户CPU时间。\\
如:在键盘中输入一句话来说,系统响应时间包括了用户从敲击键盘到文字在屏幕上显示的时间(包括了I/O时间),而用户CPU时间则是在系统中调用了输入文字的代码后代码在CPU中执行的时间。
~\\
~\\

\question \large{指令的CPI、机器的CPI、程序的CPI各能反映哪方面的性能?}

\normalsize
~\\
答:
\begin{itemize}
	\item 指令的CPI对于特定的指令而言是一个定值,反应的是指令集的性能,如x86和arm的区别
	\item 机器的CPI表示的是该机器指令集中每条指令执行平均需要多少时钟周期,反应机器或者硬件性能
	\item 程序的CPI表示该程序每条指令实行是平均需要多少时钟周期,反应代码设计优劣以及编译器的性能
\end{itemize}
 


\question \large{Example MIPS中的CPI 1.57 是如何算出来的?}

\normalsize
~\\
答:
$$
\frac{43\% \times 1 + 21\% \times 2 + 12\% \times 2 + 24\% \times 2}{1} = 1.57
$$

\end{document}